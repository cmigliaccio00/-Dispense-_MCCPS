It has been explained in the previous chapter that for a system 
\begin{equation} \label{eq:system}
	\begin{aligned}
		&x(k+1)=Ax(k)\\
		&y(k)=Cx(k)
	\end{aligned}
\end{equation}
if one collects $n$ measurements $y(i), i=1,..., n$, we can recover the state $x(k)$ at each $k$, if we are able to find $x(0)$ and then invert the equation
\begin{equation*}
	y = \mathcal{O}_n x(0)
\end{equation*}
the Theorem by Kalman, states that this is possible, that is the system is \textbf{observable}, if and only if $\text{rank}(\mathcal{O}_n)=n$.
This is the static-batch approach to the \textbf{state estimation problem}, on the other hand - as an alternative technique - if the system is observable we can estimate $x(k)$(then $x(0)$) dynamically by constructing a device called the \textbf{Observer}, in the deterministic case it is called \textbf{Luemberger Observer}.\\

\section{Review of Luemberger Observer}

A copy of the system (\ref{eq:system}) is made with the only difference of adding a correction term. Starting from this point we use $\hat{x}(k)$ to indicate the \textbf{estimate of the state at the time $k$} (discretized time), and $\hat{y}(k)$ is the output computed by using the estimate. The Luemberger Observer has the following equations: 
\begin{equation} \label{eq: Luemberger}
	\begin{aligned}
			&\hat{x}(k+1)=A\hat{x}(k)-L[\hat{y}(k)-y(k)]\\
		&\hat{y}(k)=C\hat{x}(k)
	\end{aligned}
\end{equation}
The quantity $e(k)=\hat{x}(k)-x(k)$ is the error of the estimate at time $k$, the aim is to design an online algorithm which could make $e(k)\rightarrow 0 $. \\
In order to understand the role of the matrix $L\in \mathbb{R}^{n,q}$,  called the \textbf{observer gain matrix}, we can write: 
\begin{align}
	e(k+1)&=\hat{x}(k)-x(k)=A\hat{x}(k)-L[\hat{y}(k)-y(k)]-Ax(k)=\\
	&=A\hat{x}(k)-LC\hat{x}(k)-LCx(k)-Ax(k)=\\
	&=A[\hat{x}(k)-x(k)]-LC[\hat{x}(k)-x(k)]=\\
	&=(A-LC) [\hat{x}(k)-x(k)]=\\
	&=(A-LC) \ e(k)	
\end{align}
The resulting equation $e(k+1)=(A-LC)\ e(k)$ describes an LTI discrete time dynamical system in which the state matrix is represented by $A-LC$. From the theory of dynamical systems, we know that the system is asymptotically (internally) stable if after a certain time $k$, it is verified that $e(k)\rightarrow 0$, the result we are seeking, it is verified when the \textit{eigenvalues} of $A-LC$ are in the unitary circle. \\

Regarding the matrix $A$ it is not very interesting to track $x(k)$ if it is asymptotically stable because in this situation $\lim_{k \to \infty} x(k)=0$. So $A$ is required to be stable but not asymptotically (some authors refers to this type of stability as \textbf{marginal stability}).\\

\hspace*{-5mm}
\begin{tikzpicture}
	\node [mybox] (box){%
		\begin{minipage}{.96\textwidth}     %Larghezza del box
			\textbf{Theorem} If the system is {\color{red}observable}, then there exists $L$ such that $A-LC$ is asymptotically stable. (We can drive the error to zero in order to track the state of the system).
		\end{minipage}
	};
\end{tikzpicture}

\section{State estimation by Least-squares approach}
At a certain time $k$, given the current measurement $y(k)=Cx(k)$, we might estimate $x(k)$ by solving the following problem: 
\begin{equation}
	\hat{x}(k) = \text{arg}\min_{x\in\mathbb{R}^n} \frac{1}{2} 
	\lVert y(k)-Cx \rVert_2^2 
\end{equation}
if $q>n$ and $C$ is full rank. We call $\mathcal{F}(x) =\frac{1}{2} 
\lVert y(k)-Cx \rVert_2^2$ the Least-Squares functional. 
There is a problem: the (pseudo)inversion of the matrix $C$, could be non-trivial for a medium-large dimensional problem. Even the classical Gradient Descent Algorithm would be too slow!\\
A solution is: at each $k$, we run a \textbf{single step} of gradient descent resulting in an \textbf{Online gradient descent}.\\

\hspace*{-5mm}
\begin{tikzpicture}
	\node [mybox] (box){%
		\begin{minipage}{.96\textwidth}     %Larghezza del box
				\textbf{\underline{Online Gradient Descent (OGD)}} \\
				Given the measurement $y(k)=Cx(k)$ and $\hat{x}(k)$ computed before time $k$
				\begin{align}
					\hat{x}^+(k)&=\hat{x}(k)-\tau\nabla\mathcal{F}(\hat{x}(k)) =  \hat{x}(k)-\tau C^T[C\hat{x}(k)-y(k)]\\
					&=\hat{x}(k)-\tau C^T  \ [\hat{y}(k)-y(k)] 
					\leftarrow \text{estimate of} \ x(k)\\
					\hat{x}(k+1)&=A\hat{x}^+(k) 
					\leftarrow \text{prediction of} \ x(k+1)\\
					\hat{y}(k)&=C\hat{x}(k)				
				\end{align}
		\end{minipage}
	};
\end{tikzpicture}
By merging estimate and prediction we obtain:
\begin{equation}\label{eq: OGD_obs}
	\hat{x}(k+1)=A\hat{x}(k)-\tau AC^T[\hat{y}(k)-y(k)]
\end{equation}
It can be noted that there is a certain similarity of the system (\ref{eq: OGD_obs}) and the (\ref{eq: Luemberger}), in particular the OGD is a Lumberger Observer with $L_g=\tau AC^T$.\\
Since we have fixed, in a certain way, the matrix $L_g$, one would wonder when 
\begin{equation} \label{eq:OGD_dyn}
	A-L_gC
\end{equation}
is asymptotically stable. 
We can rewrite it as $A(I-\tau C^TC)$.
If we choose $\tau<\frac{1}{\lVert C\rVert_2^2}$ then we obtain that 
$\lVert I - \tau C^TC \rVert \le 1$. Finally, two cases are to be considered: 
\begin{itemize}
	\item If $\lVert I - \tau C^TC \rVert = 1$, and A is \textit{marginally stable}, then $A-L_gC$ is marginally stable; 
	\item  If $\lVert I - \tau C^TC \rVert < 1$, and A is \textit{marginally stable}, then  $A-L_gC$ is \textbf{asymptotically stable}.
\end{itemize}

Until this moment, we have presented these results for an LTI DT dynamical system in which there are not attacks. What about \textbf{Dynamic Secure State Estimation}?

\section{Dynamic SSE with constant attack}
Recalling that a CPS under adversarial attacks on the sensors can be described by the system: 
\begin{equation}
	\begin{aligned}
		&x(k+1)=Ax(k)\\
		&y(k)=Cx(k)+a(k)
	\end{aligned}
\end{equation}
We have seen in the Second Chapter that in this case the problem of observability results in:
\begin{equation} \label{eq:CPS_attack}
	\begin{pmatrix}
		y(0)\\ \vdots \\ y(T-1)
	\end{pmatrix}=\mathcal{O}_Tx(0)+\begin{pmatrix}
	a(0) \\ \vdots \\ a(T-1)
 	\end{pmatrix}
\end{equation}
We have solved this problem for the static case in which we have seen the IST Algorithm, but in the case in which $A$ is not the identity matrix, the problem is not trivial to solve!\\
If an assumption is done on the 'shape' of the attacks the problem could be well posed, in particular we should assume that the attacks are constant and equal to a vector $a\in\mathbb{R}^q$, at this point the problem (\ref{eq:CPS_attack}) results in:
\begin{equation} 
	\begin{pmatrix}
		y(0)\\ \vdots \\ y(T-1)
	\end{pmatrix}=\mathcal{O}_nx(0)+\begin{pmatrix}
		a \\ \vdots \\ a
	\end{pmatrix} = 
	\begin{pmatrix}
		C & I\\
		CA & I \\
		\vdots & \vdots\\
		CA^{T-1} & I	
	\end{pmatrix} \begin{pmatrix}
		x(0)\\a
	\end{pmatrix}
\end{equation}
where the matrix 
\begin{equation}	\label{eq:aug_matCPS}
	\mathcal{O}_T'=\begin{pmatrix}
		C & I\\
		CA & I \\
		\vdots & \vdots\\
		CA^{T-1} & I	
	\end{pmatrix}
\end{equation}
is an \textbf{augmented observability matrix}. Is this CPS observable?\\

\noindent
In order to clarify this aspect, let us consider a couple of measurements for $k=0,1$:
\begin{align}
	&y(0)=Cx(0)+a \label{eq:first}\\ 
	&y(1)=Cx(1)+a=CAx(0)+a \label{eq:second}
\end{align}
We might manipulate algebraically these equations in order to eliminate the attack $a$ which is assumed to be constant. For example, one can subtract the (\ref{eq:second}) from the (\ref{eq:first}), and it will be obtained
\begin{align}
	&y(1)-y(0)=CAx(0) + a-Cx(0)-a=\\
	&[CA-C]x(0) =\\
	&C[A-I]x(0) \label{eq:final_equation}
\end{align} 
Moreover, let us suppose that $q=n$ so that the matrix $C[A-I]\in\mathbb{R}^{n,n}$. If such matrix would be invertible, we could recover $x(0)$ without problem by inverting the equation (\ref{eq:final_equation}). One might think that I could go further in the computation of $y(k)$ and by using such manipulations, I can eliminate the attack and recover without problems the state.\\
\underline{\textbf{BUT}} in many situations the square matrix $A-I$ is not invertible, we have seen that is reasonable that in the $\text{Spec}(A)$ (set of the eigenvalues of A) there is an eigenvalue $\lambda_i=1$ for  some $i$.
This imply that the matrix $A-I$ has a \textbf{null eigenvalue}, that is the same to confirm that the matrix is \textbf{not full rank} and for this reason \textbf{non invertible}.
One wonder if we might have a generalization of this concept. It is possible by analysing the \textbf{kernel of} $\mathcal{O}'_T$. To this aim, again, we will exploit some algebraic tricks. This time we subtract couple of rows of the matrix (\ref{eq:aug_matCPS}) from the bottom to the top, obtaining:
\begin{equation}
	\begin{pmatrix}
		C & I \\
		C(A-I) & I \\
		\vdots & \vdots \\
		CA^{T-3} (A-I) & 0\\
		CA^{T-2} (A-I) & 0
  	\end{pmatrix} \begin{pmatrix}
  		x\\a
  	\end{pmatrix}
\end{equation}
This is nothing but the system (\ref{eq:CPS_attack}) rewritten in a different form. Let us neglect at the moment the first row of the rewritten matrix. It is recognizable the matrix $\mathcal{O}_{T-1}$, however due to the fact of being multiplied by $A-I$ the linear system 
\begin{equation*}
	\mathcal{O}_{T-1}(A-I)x=0
\end{equation*}
is underdetermined and has got infinitely many solutions. Despite $\mathcal{O}_{T-1}$ is full rank (it is a minimum requirement because if the system without attack is not observable, I cannot recover the state with attacks!) we do not have a trivial kernel because of $A-I$ which is not full rank. Moreover if we add the first equation we obtain the total system
\begin{equation*}
	\begin{cases}
		(A-I)x=0\\
		Cx+a=0
	\end{cases}
\end{equation*}
We are ready to give the following proposition:\\

\hspace*{-5mm}
\begin{tikzpicture}
	\node [mybox] (box){%
		\begin{minipage}{.96\textwidth}     %Larghezza del box
			\textbf{Proposition} If the matrix $A$ has an eigenvalue = 1, the dynamic CPS with constant attack is not observable.\\
		\end{minipage}
	};
\end{tikzpicture}%

\noindent
The fact that an eigenvalue of A might be equal to one, is quite common from the moment we do not desire the situation in which the state tends to zero when $k\rightarrow\infty$.\\
Then, the proposition states that in general a CPS under attacks \textbf{is not observable} even if the attack is \textbf{constant}. However, we have not exploited yet the information about the \textbf{sparsity of the attacks} which allows us to develop a so-called \textbf{{\color{red} SPARSE OBSERVER}}. \\
Before giving the final result is useful to give a little of notation:
\begin{equation*}
	z(k) = \begin{pmatrix}
		x(k)\\a(k)
	\end{pmatrix}		\quad
	\hat{z}(k) = \begin{pmatrix}
		\hat{x}(k)\\\hat{a}(k)  
	\end{pmatrix}		\quad
	\hat{z}^{+}(k)=\begin{pmatrix}
		\hat{x}^+(k)\\\hat{a}^+(k)
	\end{pmatrix}		\quad
	G \ = \ (C \quad I)
\end{equation*}
We want to solve the problem of recover the state of the dynamic CPS under constant attacks that is to solve:
\begin{equation*}
	\min_{x\in\mathbb{R}^n, a \in \mathbb{R}^q}
	\frac{1}{2} \lVert y(k) - Gz(k) \rVert_2^2 + 
	\lambda \lVert a \rVert_1
\end{equation*}
It can be demonstrated that after a sufficient number $T$ of steps the solution is given by the following algorithm:\\

\hspace*{-5mm}
\begin{tikzpicture}
	\node [mybox] (box){%
		\begin{minipage}{.96\textwidth}     %Larghezza del box
			\textbf{\underline{SPARSE OBSERVER}}\\
			Given $y(k)=Gz(k)$ and $\hat{z}(k)$, 
			\begin{align*}
				\hat{z}^+(k)&=\hat{z}(k)-\tau G^T[G\hat{z}(k)-y(k)]
				& \leftarrow  \text{estimate of} \ z(k)\\
				\hat{x}(k+1) &= A\hat{x}^+(k) 
				 & \leftarrow \text{prediction of} x(k+1) \\
				 \hat{a}^(k+1) &= \mathbb{S}_{\tau\lambda}[\hat{a}^+(k)]
				 & \leftarrow \text{"sparsify" the attacks}\\
				 \hat{y}(k) &=G\hat{z}(k)
			\end{align*} 			
		\end{minipage}
	};
\end{tikzpicture}\\

\noindent
What are the differences from the previous version? The matrix $A$ is not the identity matrix, the state of the CPS \textbf{changes} $\Rightarrow$ in general $x(k) \ne x(k+1)$.
